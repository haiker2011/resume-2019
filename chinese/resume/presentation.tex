%-------------------------------------------------------------------------------
%	SECTION TITLE
%-------------------------------------------------------------------------------
\cvsection{演讲与分享}

\cvsubsection{部分文章}

\begin{cventries}

    \cventry
    {}
    {\href{http://www.servicemesher.com/blog/gloo-by-solo-io-is-the-first-alternative-to-istio-on-knative/}{Solo.io打造的Gloo——Knative中Istio的替代方案}}
    {ServiceMesher}
    {2019 年 5 月 16 日}
    {
      \begin{cvitems} % Description(s)
        \item {
          这篇文章主要介绍了一种 Istio 的替代方案,使用 Solo.io 公司研发的 Gloo 来替代 Istio 来使用 Knative
    \cventry
    {}
    {\href{http://www.servicemesher.com/blog/guidance-for-building-a-control-plane-for-envoy-part-4-build-for-extensibility/}{为 Envoy 构建控制面指南第4部分:构建的可扩展性}}
    {ServiceMesher}
    {2019 年 4 月 22 日}
    {
      \begin{cvitems} % Description(s)
        \item {
          这篇文章主要介绍了 Gloo 团队建议将重点放在控制平面的简单核心上,然后通过插件和微服务控制器的可组合性扩展它 
        }
      \end{cvitems}
    }

    \cventry
    {}
    {\href{http://www.servicemesher.com/blog/guidance-for-building-a-control-plane-for-envoy-part-3-domain-specific-configuration/}{为 Envoy 构建控制平面指南第3部分:领域特定配置}}
    {ServiceMesher}
    {2019 年 4 月 4 日}
    {
      \begin{cvitems} % Description(s)
        \item {
          建立最适合您的使用场景和组织架构的特定于域的配置对象和 api
        }
      \end{cvitems}
    }

    \cventry
    {}
    {\href{http://www.servicemesher.com/blog/building-a-control-plane-for-envoy/}{为 Envoy 赋能——如何基于 Envoy 构建一个多用途控制平面}}
    {ServiceMesher}
    {2019 年 3 月 25 日}
    {
      \begin{cvitems} % Description(s)
        \item {
          在本文章阐述了 Envoy 的工作原理、为什么要选择 Envoy 以及在构建一个控制平面过程中要做出的技术体系结构的抉择
        }
      \end{cvitems}
    }

    \cventry
    {}
    {\href{http://www.servicemesher.com/blog/knative-whittling-down-the-code/}{Knative:精简代码之道}}
    {ServiceMesher}
    {2019 年 3 月 1 日}
    {
      \begin{cvitems} % Description(s)
        \item 本文直观地向我们展示了如何使用Knative来一步一步逐渐精简我们的代码,来更加简单容易的开发云原生应用
      \end{cvitems}
    }
  
\end{cventries}

% \cvsubsection{技术演讲}

% %-------------------------------------------------------------------------------
% %	CONTENT
% %-------------------------------------------------------------------------------
% \begin{cventries}

% %---------------------------------------------------------
%   \cventry
%     {KubeCon China 2018}
%     {\href{http://sched.co/FvLV}{对 Kubeflow 上的机器学习工作负载做基准测试}}
%     {中国上海} % Location
%     {2018 年 11 月} % Date(s)
%     {
%       \begin{cvitems} % Description(s)
%         \item {
%           在本次演讲中我们介绍基于 Kubeflow 的开源基准化工具 Kubebench,其帮助我们通过自动化和一致的规范,更好的理解 Kubernetes 上的 ML 工作量的性能特征。我们还说明我们可以怎样利用来自学术界和工业界的其他基准化成就,如 MLPerf 和 Dawnbench
%         }
%       \end{cvitems}
%     }

%   \cventry
%     {统计之都 2018 年 Meetup}
%     {\href{https://docs.google.com/presentation/d/1ED24TCnlBVzyJz0aCEAtXQQh0_W1RKSeapP3QZ0fTKA/edit?usp=sharing}{Kubeflow: Run ML workloads on Kubernetes}}
%     {中国上海} % Location
%     {2018 年 7 月} % Date(s)
%     {
%     }

%   \cventry
%     {第十届中国 R 语言会议}
%     {\href{http://slides.com/gaocegege/processing-r}{
%       Processing.R: 使用 R 语言实现新媒体艺术作品}}
%     {中国上海} % Location
%     {2017 年 12 月} % Date(s)
%     {
%       \begin{cvitems} % Description(s)
%         \item {
%           本次演讲介绍了 Processing.R。通过这一项目,用户可以利用 R 语言进行新媒体艺术作品的创作
%         }
%       \end{cvitems}
%     }

%   \cventry
%     {2017 年东岳技术分享}
%     {\href{https://docs.google.com/presentation/d/1ylRT4VvydWbR7SyTQzNZOLpkXtgSZJiEl5nmXY1KuJw/edit?usp=sharing}{Processing.R 设计,实现与使用}}
%     {中国上海} % Location
%     {2017 年 7 月} % Date(s)
%     {
%     }

%   \cventry
%     {2016 年东岳技术分享}
%     {\href{https://docs.google.com/presentation/d/1Ru4Dm9TLoyxnJgFqvsCHrb82VT622H-zBSgAe1vJL44/edit?usp=sharing}{Docker 入门介绍}}
%     {中国上海} % Location
%     {2016 年 7 月} % Date(s)
%     {
%     }

% \cvsubsection{技术演讲}

% %-------------------------------------------------------------------------------
% %	CONTENT
% %-------------------------------------------------------------------------------
% \begin{cventries}

% %---------------------------------------------------------
%   \cventry
%     {KubeCon China 2018}
%     {\href{http://sched.co/FvLV}{对 Kubeflow 上的机器学习工作负载做基准测试}}
%     {中国上海} % Location
%     {2018 年 11 月} % Date(s)
%     {
%       \begin{cvitems} % Description(s)
%         \item {
%           在本次演讲中我们介绍基于 Kubeflow 的开源基准化工具 Kubebench,其帮助我们通过自动化和一致的规范,更好的理解 Kubernetes 上的 ML 工作量的性能特征。我们还说明我们可以怎样利用来自学术界和工业界的其他基准化成就,如 MLPerf 和 Dawnbench
%         }
%       \end{cvitems}
%     }

%   \cventry
%     {统计之都 2018 年 Meetup}
%     {\href{https://docs.google.com/presentation/d/1ED24TCnlBVzyJz0aCEAtXQQh0_W1RKSeapP3QZ0fTKA/edit?usp=sharing}{Kubeflow: Run ML workloads on Kubernetes}}
%     {中国上海} % Location
%     {2018 年 7 月} % Date(s)
%     {
%     }

%   \cventry
%     {第十届中国 R 语言会议}
%     {\href{http://slides.com/gaocegege/processing-r}{
%       Processing.R: 使用 R 语言实现新媒体艺术作品}}
%     {中国上海} % Location
%     {2017 年 12 月} % Date(s)
%     {
%       \begin{cvitems} % Description(s)
%         \item {
%           本次演讲介绍了 Processing.R。通过这一项目,用户可以利用 R 语言进行新媒体艺术作品的创作
%         }
%       \end{cvitems}
%     }

%   \cventry
%     {2017 年东岳技术分享}
%     {\href{https://docs.google.com/presentation/d/1ylRT4VvydWbR7SyTQzNZOLpkXtgSZJiEl5nmXY1KuJw/edit?usp=sharing}{Processing.R 设计,实现与使用}}
%     {中国上海} % Location
%     {2017 年 7 月} % Date(s)
%     {
%     }

%   \cventry
%     {2016 年东岳技术分享}
%     {\href{https://docs.google.com/presentation/d/1Ru4Dm9TLoyxnJgFqvsCHrb82VT622H-zBSgAe1vJL44/edit?usp=sharing}{Docker 入门介绍}}
%     {中国上海} % Location
%     {2016 年 7 月} % Date(s)
%     {
%     }

% %---------------------------------------------------------
% \end{cventries}
