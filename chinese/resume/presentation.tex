%-------------------------------------------------------------------------------
%	SECTION TITLE
%-------------------------------------------------------------------------------
\cvsection{演讲与分享}

\cvsubsection{部分文章}

\begin{cventries}
  \cventry
    {}
    {\href{https://github.com/gaocegege/papers-notebook}{论文阅读笔记}}
    {GitHub}
    {2016 年 11 月至今}
    {
      \begin{cvitems} % Description(s)
        \item 记录了 100 余篇学术论文的阅读笔记,论文的方向多为分布式系统,虚拟化,安全,机器学习,超参数训练,网络模型结构搜索等领域
      \end{cvitems}
    }

  \cventry
    {}
    {\href{http://t.cn/Eh7yABv}{机器学习平台漫谈}}
    {\href{http://gaocegege.com}{gaocegege.com}}
    {2018 年 7 月 24 日}
    {
      \begin{cvitems} % Description(s)
        \item {
          随着深度学习的兴起,机器学习在最近几年以星火燎原之势席卷了整个科技行业。而在整个机器学习的工作流中,模型训练的代码只是其中的一小部分。除此之外,训练任务的监控,日志的回收,超参数的选择与优化,模型的发布与集成,数据清洗,特征提取等等,都是流程中不可或缺的部分。因此,有一些工具和公司的产品,致力于为机器学习从业者提供一个统一的平台,帮助用户更好地完成其机器学习业务的落地。这篇文章是关于机器学习平台产品的分析对比,由于利益相关性只放出国外的产品
        }
      \end{cvitems}
    }

  \cventry
    {}
    {\href{http://gaocegege.com/Blog/kubernetes/operator}{Kubernetes CRD Operator 实现指南}}
    {\href{http://gaocegege.com}{gaocegege.com}}
    {2018 年 6 月 11 日}
    {
      \begin{cvitems} % Description(s)
        \item {
          Kubernetes 已经成为了集群调度领域最炙手可热的开源项目之一。而多工作负载支持,是讨论到集群调度时不得不谈的一个话题。CRD 是 Kubernetes 的一个特性,通过它,集群可以支持自定义的资源类型,这是在 Kubernetes 集群上支持多工作负载的方式之一。本文希望讨论在实现一个 Kubernetes CRD Operator 时可能遇到的问题以及解决方案,抛砖引玉,探索实现的最佳实践
        }
      \end{cvitems}
    }

  \cventry
    {}
    {\href{http://t.cn/Eh7UCQx}{Katib: Kubernetes native 的超参数训练系统}}
    {\href{http://gaocegege.com}{gaocegege.com}}
    {2018 年 3 月 7 日}
    {
      \begin{cvitems} % Description(s)
        \item {
          这篇文章主要介绍了 Katib,一个由 NTT 贡献到 Kubeflow 社区的超参数训练系统。面向人群为对在 Kubernetes 上运行机器学习负载感兴趣的同学。
        }
      \end{cvitems}
    }
\end{cventries}

% \cvsubsection{技术演讲}

% %-------------------------------------------------------------------------------
% %	CONTENT
% %-------------------------------------------------------------------------------
% \begin{cventries}

% %---------------------------------------------------------
%   \cventry
%     {KubeCon China 2018}
%     {\href{http://sched.co/FvLV}{对 Kubeflow 上的机器学习工作负载做基准测试}}
%     {中国上海} % Location
%     {2018 年 11 月} % Date(s)
%     {
%       \begin{cvitems} % Description(s)
%         \item {
%           在本次演讲中我们介绍基于 Kubeflow 的开源基准化工具 Kubebench,其帮助我们通过自动化和一致的规范,更好的理解 Kubernetes 上的 ML 工作量的性能特征。我们还说明我们可以怎样利用来自学术界和工业界的其他基准化成就,如 MLPerf 和 Dawnbench
%         }
%       \end{cvitems}
%     }

%   \cventry
%     {统计之都 2018 年 Meetup}
%     {\href{https://docs.google.com/presentation/d/1ED24TCnlBVzyJz0aCEAtXQQh0_W1RKSeapP3QZ0fTKA/edit?usp=sharing}{Kubeflow: Run ML workloads on Kubernetes}}
%     {中国上海} % Location
%     {2018 年 7 月} % Date(s)
%     {
%     }

%   \cventry
%     {第十届中国 R 语言会议}
%     {\href{http://slides.com/gaocegege/processing-r}{
%       Processing.R: 使用 R 语言实现新媒体艺术作品}}
%     {中国上海} % Location
%     {2017 年 12 月} % Date(s)
%     {
%       \begin{cvitems} % Description(s)
%         \item {
%           本次演讲介绍了 Processing.R。通过这一项目,用户可以利用 R 语言进行新媒体艺术作品的创作
%         }
%       \end{cvitems}
%     }

%   \cventry
%     {2017 年东岳技术分享}
%     {\href{https://docs.google.com/presentation/d/1ylRT4VvydWbR7SyTQzNZOLpkXtgSZJiEl5nmXY1KuJw/edit?usp=sharing}{Processing.R 设计,实现与使用}}
%     {中国上海} % Location
%     {2017 年 7 月} % Date(s)
%     {
%     }

%   \cventry
%     {2016 年东岳技术分享}
%     {\href{https://docs.google.com/presentation/d/1Ru4Dm9TLoyxnJgFqvsCHrb82VT622H-zBSgAe1vJL44/edit?usp=sharing}{Docker 入门介绍}}
%     {中国上海} % Location
%     {2016 年 7 月} % Date(s)
%     {
%     }

% %---------------------------------------------------------
% \end{cventries}
